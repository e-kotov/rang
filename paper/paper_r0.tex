% Options for packages loaded elsewhere
\PassOptionsToPackage{unicode}{hyperref}
\PassOptionsToPackage{hyphens}{url}
%


\PassOptionsToPackage{table}{xcolor}

\documentclass[
  10pt,
  letterpaper,
]{article}

\usepackage{amsmath,amssymb}
\usepackage{lmodern}
\usepackage{iftex}
\ifPDFTeX
  \usepackage[T1]{fontenc}
  \usepackage[utf8]{inputenc}
  \usepackage{textcomp} % provide euro and other symbols
\else % if luatex or xetex
  \usepackage{unicode-math}
  \defaultfontfeatures{Scale=MatchLowercase}
  \defaultfontfeatures[\rmfamily]{Ligatures=TeX,Scale=1}
\fi
% Use upquote if available, for straight quotes in verbatim environments
\IfFileExists{upquote.sty}{\usepackage{upquote}}{}
\IfFileExists{microtype.sty}{% use microtype if available
  \usepackage[]{microtype}
  \UseMicrotypeSet[protrusion]{basicmath} % disable protrusion for tt fonts
}{}
\makeatletter
\@ifundefined{KOMAClassName}{% if non-KOMA class
  \IfFileExists{parskip.sty}{%
    \usepackage{parskip}
  }{% else
    \setlength{\parindent}{0pt}
    \setlength{\parskip}{6pt plus 2pt minus 1pt}}
}{% if KOMA class
  \KOMAoptions{parskip=half}}
\makeatother
\usepackage{xcolor}
\usepackage[top=0.85in,left=2.75in,footskip=0.75in]{geometry}
\setlength{\emergencystretch}{3em} % prevent overfull lines
\setcounter{secnumdepth}{-\maxdimen} % remove section numbering

\usepackage{color}
\usepackage{fancyvrb}
\newcommand{\VerbBar}{|}
\newcommand{\VERB}{\Verb[commandchars=\\\{\}]}
\DefineVerbatimEnvironment{Highlighting}{Verbatim}{commandchars=\\\{\}}
% Add ',fontsize=\small' for more characters per line
\usepackage{framed}
\definecolor{shadecolor}{RGB}{241,243,245}
\newenvironment{Shaded}{\begin{snugshade}}{\end{snugshade}}
\newcommand{\AlertTok}[1]{\textcolor[rgb]{0.68,0.00,0.00}{#1}}
\newcommand{\AnnotationTok}[1]{\textcolor[rgb]{0.37,0.37,0.37}{#1}}
\newcommand{\AttributeTok}[1]{\textcolor[rgb]{0.40,0.45,0.13}{#1}}
\newcommand{\BaseNTok}[1]{\textcolor[rgb]{0.68,0.00,0.00}{#1}}
\newcommand{\BuiltInTok}[1]{\textcolor[rgb]{0.00,0.23,0.31}{#1}}
\newcommand{\CharTok}[1]{\textcolor[rgb]{0.13,0.47,0.30}{#1}}
\newcommand{\CommentTok}[1]{\textcolor[rgb]{0.37,0.37,0.37}{#1}}
\newcommand{\CommentVarTok}[1]{\textcolor[rgb]{0.37,0.37,0.37}{\textit{#1}}}
\newcommand{\ConstantTok}[1]{\textcolor[rgb]{0.56,0.35,0.01}{#1}}
\newcommand{\ControlFlowTok}[1]{\textcolor[rgb]{0.00,0.23,0.31}{#1}}
\newcommand{\DataTypeTok}[1]{\textcolor[rgb]{0.68,0.00,0.00}{#1}}
\newcommand{\DecValTok}[1]{\textcolor[rgb]{0.68,0.00,0.00}{#1}}
\newcommand{\DocumentationTok}[1]{\textcolor[rgb]{0.37,0.37,0.37}{\textit{#1}}}
\newcommand{\ErrorTok}[1]{\textcolor[rgb]{0.68,0.00,0.00}{#1}}
\newcommand{\ExtensionTok}[1]{\textcolor[rgb]{0.00,0.23,0.31}{#1}}
\newcommand{\FloatTok}[1]{\textcolor[rgb]{0.68,0.00,0.00}{#1}}
\newcommand{\FunctionTok}[1]{\textcolor[rgb]{0.28,0.35,0.67}{#1}}
\newcommand{\ImportTok}[1]{\textcolor[rgb]{0.00,0.46,0.62}{#1}}
\newcommand{\InformationTok}[1]{\textcolor[rgb]{0.37,0.37,0.37}{#1}}
\newcommand{\KeywordTok}[1]{\textcolor[rgb]{0.00,0.23,0.31}{#1}}
\newcommand{\NormalTok}[1]{\textcolor[rgb]{0.00,0.23,0.31}{#1}}
\newcommand{\OperatorTok}[1]{\textcolor[rgb]{0.37,0.37,0.37}{#1}}
\newcommand{\OtherTok}[1]{\textcolor[rgb]{0.00,0.23,0.31}{#1}}
\newcommand{\PreprocessorTok}[1]{\textcolor[rgb]{0.68,0.00,0.00}{#1}}
\newcommand{\RegionMarkerTok}[1]{\textcolor[rgb]{0.00,0.23,0.31}{#1}}
\newcommand{\SpecialCharTok}[1]{\textcolor[rgb]{0.37,0.37,0.37}{#1}}
\newcommand{\SpecialStringTok}[1]{\textcolor[rgb]{0.13,0.47,0.30}{#1}}
\newcommand{\StringTok}[1]{\textcolor[rgb]{0.13,0.47,0.30}{#1}}
\newcommand{\VariableTok}[1]{\textcolor[rgb]{0.07,0.07,0.07}{#1}}
\newcommand{\VerbatimStringTok}[1]{\textcolor[rgb]{0.13,0.47,0.30}{#1}}
\newcommand{\WarningTok}[1]{\textcolor[rgb]{0.37,0.37,0.37}{\textit{#1}}}

\providecommand{\tightlist}{%
  \setlength{\itemsep}{0pt}\setlength{\parskip}{0pt}}\usepackage{longtable,booktabs,array}
\usepackage{calc} % for calculating minipage widths
% Correct order of tables after \paragraph or \subparagraph
\usepackage{etoolbox}
\makeatletter
\patchcmd\longtable{\par}{\if@noskipsec\mbox{}\fi\par}{}{}
\makeatother
% Allow footnotes in longtable head/foot
\IfFileExists{footnotehyper.sty}{\usepackage{footnotehyper}}{\usepackage{footnote}}
\makesavenoteenv{longtable}
\usepackage{graphicx}
\makeatletter
\def\maxwidth{\ifdim\Gin@nat@width>\linewidth\linewidth\else\Gin@nat@width\fi}
\def\maxheight{\ifdim\Gin@nat@height>\textheight\textheight\else\Gin@nat@height\fi}
\makeatother
% Scale images if necessary, so that they will not overflow the page
% margins by default, and it is still possible to overwrite the defaults
% using explicit options in \includegraphics[width, height, ...]{}
\setkeys{Gin}{width=\maxwidth,height=\maxheight,keepaspectratio}
% Set default figure placement to htbp
\makeatletter
\def\fps@figure{htbp}
\makeatother

% Use adjustwidth environment to exceed column width (see example table in text)
\usepackage{changepage}

% marvosym package for additional characters
\usepackage{marvosym}

% cite package, to clean up citations in the main text. Do not remove.
% Using natbib instead
% \usepackage{cite}

% Use nameref to cite supporting information files (see Supporting Information section for more info)
\usepackage{nameref,hyperref}

% line numbers
\usepackage[right]{lineno}

% ligatures disabled
\usepackage{microtype}
\DisableLigatures[f]{encoding = *, family = * }

% create "+" rule type for thick vertical lines
\newcolumntype{+}{!{\vrule width 2pt}}

% create \thickcline for thick horizontal lines of variable length
\newlength\savedwidth
\newcommand\thickcline[1]{%
  \noalign{\global\savedwidth\arrayrulewidth\global\arrayrulewidth 2pt}%
  \cline{#1}%
  \noalign{\vskip\arrayrulewidth}%
  \noalign{\global\arrayrulewidth\savedwidth}%
}

% \thickhline command for thick horizontal lines that span the table
\newcommand\thickhline{\noalign{\global\savedwidth\arrayrulewidth\global\arrayrulewidth 2pt}%
\hline
\noalign{\global\arrayrulewidth\savedwidth}}

% Text layout
\raggedright
\setlength{\parindent}{0.5cm}
\textwidth 5.25in 
\textheight 8.75in

% Bold the 'Figure #' in the caption and separate it from the title/caption with a period
% Captions will be left justified
\usepackage[aboveskip=1pt,labelfont=bf,labelsep=period,justification=raggedright,singlelinecheck=off]{caption}
\renewcommand{\figurename}{Fig}

% Remove brackets from numbering in List of References
\makeatletter
\renewcommand{\@biblabel}[1]{\quad#1.}
\makeatother

% Header and Footer with logo
\usepackage{lastpage,fancyhdr}
\usepackage{epstopdf}
%\pagestyle{myheadings}
\pagestyle{fancy}
\fancyhf{}
%\setlength{\headheight}{27.023pt}
%\lhead{\includegraphics[width=2.0in]{PLOS-submission.eps}}
\rfoot{\thepage/\pageref{LastPage}}
\renewcommand{\headrulewidth}{0pt}
\renewcommand{\footrule}{\hrule height 2pt \vspace{2mm}}
\fancyheadoffset[L]{2.25in}
\fancyfootoffset[L]{2.25in}
\lfoot{\today}
\makeatletter
\makeatother
\makeatletter
\makeatother
\makeatletter
\@ifpackageloaded{caption}{}{\usepackage{caption}}
\AtBeginDocument{%
\ifdefined\contentsname
  \renewcommand*\contentsname{Table of contents}
\else
  \newcommand\contentsname{Table of contents}
\fi
\ifdefined\listfigurename
  \renewcommand*\listfigurename{List of Figures}
\else
  \newcommand\listfigurename{List of Figures}
\fi
\ifdefined\listtablename
  \renewcommand*\listtablename{List of Tables}
\else
  \newcommand\listtablename{List of Tables}
\fi
\ifdefined\figurename
  \renewcommand*\figurename{Figure}
\else
  \newcommand\figurename{Figure}
\fi
\ifdefined\tablename
  \renewcommand*\tablename{Table}
\else
  \newcommand\tablename{Table}
\fi
}
\@ifpackageloaded{float}{}{\usepackage{float}}
\floatstyle{ruled}
\@ifundefined{c@chapter}{\newfloat{codelisting}{h}{lop}}{\newfloat{codelisting}{h}{lop}[chapter]}
\floatname{codelisting}{Listing}
\newcommand*\listoflistings{\listof{codelisting}{List of Listings}}
\makeatother
\makeatletter
\@ifpackageloaded{caption}{}{\usepackage{caption}}
\@ifpackageloaded{subcaption}{}{\usepackage{subcaption}}
\makeatother
\makeatletter
\makeatother
\ifLuaTeX
  \usepackage{selnolig}  % disable illegal ligatures
\fi
\usepackage[numbers,square,comma]{natbib}
\bibliographystyle{plos2015}
\IfFileExists{bookmark.sty}{\usepackage{bookmark}}{\usepackage{hyperref}}
\IfFileExists{xurl.sty}{\usepackage{xurl}}{} % add URL line breaks if available
\urlstyle{same} % disable monospaced font for URLs
\hypersetup{
  pdftitle={rang: Reconstructing reproducible R computational environments},
  pdfkeywords={R, reproducibility, docker},
  hidelinks,
  pdfcreator={LaTeX via pandoc}}



\begin{document}
\vspace*{0.2in}

% Title must be 250 characters or less.
\begin{flushleft}
{\Large
\textbf\newline{rang: Reconstructing reproducible R computational
environments} % Please use "sentence case" for title and headings (capitalize only the first word in a title (or heading), the first word in a subtitle (or subheading), and any proper nouns).
}
\newline
\\
% Insert author names, affiliations and corresponding author email (do not include titles, positions, or degrees).
Chung-hong Chan\textsuperscript{1*}, David Schoch\textsuperscript{1}
\\
\bigskip
\textbf{1} GESIS Leibniz-Institut für Sozialwissenschaften, 
\bigskip


% Use the asterisk to denote corresponding authorship and provide email address in note below.
* chung-hong.chan@gesis.org

\end{flushleft}

\section*{Abstract}
A complete declarative description of the computational environment is
often missing when researchers share their materials. Without such
description, software obsolescence and missing system components can
jeopardize computational reproducibility in the future, even when data
and computer code are available. The R package rang is a complete
solution for generating the declarative description for other
researchers to automatically reconstruct the computational environment
at a specific time point. The reconstruction process, based on Docker,
has been tested for R code as old as 2001. The declarative description
generated by rang satisfies the definition of a reproducible research
compendium and can be shared as such. In this contribution, we show how
rang can be used to make otherwise unexecutable code, spanning from
fields such as computational social science and bioinformatics,
executable again. We also provide instructions on how to use rang to
construct reproducible and shareable research compendia of current
research. The package is currently available from CRAN
(https://cran.r-project.org/web/packages/rang/index.html) and GitHub
(https://github.com/chainsawriot/rang).


\linenumbers\hypertarget{background}{%
\section{Background}\label{background}}

\emph{``In some cases the polarization estimation will not work \ldots{}
This is \emph{NOT} a problem in the method, it is entirely dependent on
the numpy version (and even the OS's). If you have different versions of
numpy or even the same version of numpy on a different OS configuration,
different networks will fail randomly\ldots{} {[}F{]}or instance, the
109th Congress will fail, but will work entirely normally on a different
numpy version, which will fail on a different Congress network.''}

\begin{quote}
\textasciitilde{} excerpt of
\href{https://www.michelecoscia.com/?page_id=2105}{the README file} of a
software for polarization estimation
\end{quote}

Other than bad programming practices \citep{trisovic:2022}, the main
computing barrier to computational reproducibility is the failure to
reconstruct the computational environment like the one used by the
original researchers. This task looks trivially simple. But as computer
science research has shown, this task is incredibly complex
\citep{abate:2015:MCR, dolstra:2010:N}. For a usual scripting language
such as R \footnote{In this paper, we will focus on R, a popular
  programming language used frequently in various computational fields
  (e.g.~computational social science, bioinformatics).}, that pertains
to four aspects: A) operating system, B) system components such as
\texttt{libxml2}, C) the exact R version, and D) what and which version
of the installed R packages. We will call them Component A, B, C, D in
the following sections. Any change in these four components can possibly
affect the execution of any shared computer code. For example, the lack
of the system component \texttt{libxml2} can impact whether the R
package \texttt{xml2} can be installed on a Linux system. If the shared
computer code requires the R package \texttt{xml2}, then the whole
execution fails.

In reality, the impact of Component A is relatively weak as mainstream,
open source programming languages and their software libraries are
usually cross platform. In modern computational research, Linux is the
de-facto operating system in high performance computing environments
(e.g.~Slurm). Instead, the impact of Components B, C, and D is much
higher. Component D is the most volatile among them all as there are
many possible combinations of R packages and versions. Software updates
with breaking changes (even in a dependency) might render existing
shared code using those changed features not executable or not producing
the same result anymore. Also, software obsolescence is commonplace,
especially since academic software is often not well maintained due to
lack of incentives \citep{merow:2023:B}.

The DevOps (software development and IT operations) community is also
confronted with this problem. The issue is usually half-jokingly
referred to as ``it works on my machine''-problem \citep[a software
works on someone's local machine but is not working anymore when
deployed to the production system, indicates the software tacitly
depends on the computational environment of the local
machine]{valstar:2020:UDS}. A partial solution to this problem from the
DevOps community is called \emph{containerization}. In essence, to
containerize is to develop and deploy the software together with all the
libraries and the operating system in an OS-level virtualization
environment. In this way, software dependency issues can be resolved
inside the isolated virtualized software environment and independent of
what is installed on the local computer. Docker is a popular choice in
the DevOps world for containerization.

To build a container, one needs to write a plain text declarative
description of the required computational environment. Inside this
declarative description, it should pin down all four Components
mentioned above. For Docker, it is in the form of a plain text file
called \texttt{Dockerfile}. This \texttt{Dockerfile} is then used as the
recipe to build a Docker image, where the four Components are assembled.
Then, one can launch a container with the built Docker image.

There has been many papers written on how containerization solutions
such as Docker can be helpful also to foster computational
reproducibility of science
\citep[e.g.][]{nuest:2019, peikert:2021:RDA, boettiger:2017:IR}.
Although tutorials are available \citep[e.g.][]{nuest:2019}, providing a
declarative description of the computational environment in the form of
\texttt{Dockerfile} is far from the standard code sharing practice. This
might be due to a lack of (DevOps) skills of most scientists to create a
\texttt{Dockerfile} \citep{kim:2018:E}. But there are many tools
available to automate the process \citep[e.g.][]{nuest:2019}. The case
in point described in this paper, \texttt{rang}, is one of them. We
argue that \texttt{rang} is the only easy-to-use solution available that
can pin down and restore all four components without the reliance on any
commercial service .

\hypertarget{existing-solutions}{%
\subsection{Existing solutions}\label{existing-solutions}}

\texttt{renv} \citep{renvrpkg} (and its derivatives such as
\texttt{jetpack} and its predecessor \texttt{packrat}) takes a similar
approach to Python's \texttt{virtualenv} and Ruby's \texttt{Gem} to pin
down the exact version of R packages using a ``lock file''. Other
solutions such as \texttt{checkpoint} \citep{checkpointrpkg} depend on
the availability of The Microsoft R Application Network (MRAN, a
time-stamped daily backup of CRAN), which will be shut down on July 1st,
2023. \texttt{groundhog} \citep{groundhogrpkg} used to depend on MRAN
but has a plan to switch to their home-grown R package repository. These
solution can effectively pin down Component C and D. But they can only
restore component D. Also, for solutions depending on MRAN, there is a
limit on how far back this reproducibility can go, since MRAN can only
go back as far as September 17, 2014. Additionally, it only covers CRAN
packages.

\texttt{containerit} \citep{nuest:2019} takes the current state of the
computational environment and documents it as a \texttt{Dockerfile}.
\texttt{containerit} makes the assumption that Component A has a weak
influence on computational reproducibility and therefore defaults to
Linux-based Rocker base images \citep{boettiger:2017:IR}. In this way,
it fixes Component A. But \texttt{containerit} does not pin down the
exact version of R packages. Therefore, it can pin down components A, B,
C, but only a part of component D. \texttt{dockta} \footnote{\url{https://stencila.github.io/dockta/}}
is another containerization solution that can potentially pin down all
components due to the fact that MRAN is used. But it also suffers from
the same limitations mentioned above.

It is also worth mentioning that MRAN is not the only archival service.
Posit also provides a free (\emph{gratis}) time-stamped daily backup of
CRAN and Bioconductor (a series of repositories of R package for
bioinformatics and computational biology) called Posit Public Package
Manager \footnote{\url{https://packagemanager.rstudio.com/client/\#/repos/2/packages/}}.
It can goes as far back as October 10, 2017.

These solutions are better for prospective usage, i.e.~using them now to
ensure the reproducibility of the current research for future
researchers. \texttt{rang} mostly targets retrospective usage,
i.e.~using \texttt{rang} to reconstruct historical R computational
environments for which the declarative descriptions are not available.
One can think of \texttt{rang} as an archaeological tool. In this realm,
we could not find any existing solution targeting R specifically which
does not currently depend on MRAN.

\hypertarget{basic-usage}{%
\section{Basic usage}\label{basic-usage}}

There are two important functions of \texttt{rang}: \texttt{resolve()}
and \texttt{dockerize()}.

\texttt{resolve()} queries various web services from the r-hub project
of the R Consortium for information about R packages at a specific time
point that is necessary for reconstructing a computational environment,
e.g.~(deep) dependencies (Component D), R version (Component C), and
system requirements (Component B). For instance, if there was a
computational environment constructed on 2020-01-16 (called ``snapshot
date'') with the several natural language processing R packages,
\texttt{resolve()} can be used to resolve all the dependencies of these
R packages. Currently, \texttt{rang} supports CRAN, Bioconductor,
GitHub, and local packages.

\begin{Shaded}
\begin{Highlighting}[]
\FunctionTok{library}\NormalTok{(rang)}
\NormalTok{graph }\OtherTok{\textless{}{-}} \FunctionTok{resolve}\NormalTok{(}\AttributeTok{pkgs =} \FunctionTok{c}\NormalTok{(}\StringTok{"openNLP"}\NormalTok{, }\StringTok{"LDAvis"}\NormalTok{, }\StringTok{"topicmodels"}\NormalTok{, }\StringTok{"quanteda"}\NormalTok{),}
                 \AttributeTok{snapshot\_date =} \StringTok{"2020{-}01{-}16"}\NormalTok{)}
\NormalTok{graph}
\end{Highlighting}
\end{Shaded}

The resolved result is an S3 object called \texttt{rang}. The
information contained in a \texttt{rang} object can then be used to
construct a computational environment in a similar manner as
\texttt{containerit}, but with the packages and R versions pinned on the
snapshot date. Then, the function \texttt{dockerize()} is used to
generate the \texttt{Dockerfile} and other scripts in the
\texttt{output\_dir}.

\begin{Shaded}
\begin{Highlighting}[]
\FunctionTok{dockerize}\NormalTok{(graph, }\AttributeTok{output\_dir =} \StringTok{"docker"}\NormalTok{)}
\end{Highlighting}
\end{Shaded}

For R \textgreater= 3.1, the base images from the Rocker project are
used \citep{boettiger:2017:IR}. For R \textless{} 3.1 but \textgreater=
1.3.1, a custom base image based on Debian is used. As of writing,
\texttt{rang} does not support R \textless{} 1.3.1, i.e.~snapshot date
earlier than 2001-08-31 (which is 13 years earlier than all solutions
depending on MRAN). There are two features of \texttt{dockerize()} that
are important for future reproducibility.

\begin{enumerate}
\def\labelenumi{\arabic{enumi}.}
\tightlist
\item
  By default, the Docker image building process downloads source
  packages from their sources and then compiles them. This step depends
  on the future availability of R packages on CRAN (which is extremely
  likely to be the case in the near future, given the continuous
  availability since 1997-04-23) \footnote{\url{https://stat.ethz.ch/pipermail/r-announce/1997/000001.html}},
  Bioconductor, and Github. However, it is also possible to cache (or
  archive) the source packages now. The archived R packages can then be
  used instead during the building process. The significance of this
  step in terms of long-term computational reproducibility will be
  discussed in Section 4.
\end{enumerate}

\begin{Shaded}
\begin{Highlighting}[]
\FunctionTok{dockerize}\NormalTok{(graph, }\AttributeTok{output\_dir =} \StringTok{"docker"}\NormalTok{, }\AttributeTok{cache =} \ConstantTok{TRUE}\NormalTok{)}
\end{Highlighting}
\end{Shaded}

\begin{enumerate}
\def\labelenumi{\arabic{enumi}.}
\setcounter{enumi}{1}
\tightlist
\item
  It is also possible to install R packages in a separate library during
  the building process to isolate all these R packages from the main
  library.
\end{enumerate}

\begin{Shaded}
\begin{Highlighting}[]
\FunctionTok{dockerize}\NormalTok{(graph, }\AttributeTok{output\_dir =} \StringTok{"docker"}\NormalTok{, }\AttributeTok{cache =} \ConstantTok{TRUE}\NormalTok{,}
          \AttributeTok{lib =} \StringTok{"anotherlibrary"}\NormalTok{)}
\end{Highlighting}
\end{Shaded}

For the sake of completeness, the instructions for building the Docker
image and running the Docker container on Unix-like systems are included
here.

\begin{Shaded}
\begin{Highlighting}[]
\BuiltInTok{cd}\NormalTok{ docker}
\CommentTok{\#\# might need to sudo}
\ExtensionTok{docker}\NormalTok{ build }\AttributeTok{{-}t}\NormalTok{ rangimg .}
\CommentTok{\#\# interactive environment}
\ExtensionTok{docker}\NormalTok{ run }\AttributeTok{{-}{-}rm} \AttributeTok{{-}{-}name} \StringTok{"rangcontainer"} \AttributeTok{{-}ti}\NormalTok{ rangimg}
\end{Highlighting}
\end{Shaded}

\hypertarget{project-scanning}{%
\subsection{Project scanning}\label{project-scanning}}

The first argument of \texttt{resolve()} is processed by a separate
function called \texttt{as\_pkgrefs()}. For interoperability,
\texttt{rang} supports the ``package references'' standard \footnote{\url{https://r-lib.github.io/pkgdepends/reference/pkg_refs.html}}
used also in other packages such as \texttt{renv} \citep{renvrpkg}. It
is mostly used for converting ``shorthands'' (e.g.~\texttt{xml2} and
\texttt{S4Vectors}) to package references (e.g.~\texttt{cran::xml2} and
\texttt{bioc::S4Vectors}).

When \texttt{as\_pkgrefs()} is applied to a single path of a directory,
it scans all relevant files (\texttt{DESCRIPTION}, R scripts and R
Markdown files) for all R packages used (based on
\texttt{renv::dependencies()} ). How it works is demonstrated in three
of the following examples below. But an important caveat is that it can
only scan CRAN and Bioconductor packages.

\hypertarget{case-studies}{%
\section{Case Studies}\label{case-studies}}

The following are some examples of how \texttt{rang} can be used to make
shared, but otherwise unexecutable, R code runnable again. The examples
were drawn from various fields spanning from political science,
psychological science, and bioinformatics.

\hypertarget{quanteda-joss-paper}{%
\subsection{quanteda JOSS paper}\label{quanteda-joss-paper}}

The software paper of the text analysis R package \texttt{quanteda} was
published on 2018-10-06 \citep{benoit:2018}. In the paper, the following
R code snippet is included.

\begin{Shaded}
\begin{Highlighting}[]
\FunctionTok{library}\NormalTok{(}\StringTok{"quanteda"}\NormalTok{)}
\CommentTok{\# construct the feature co{-}occurrence matrix}
\NormalTok{examplefcm }\OtherTok{\textless{}{-}}
\FunctionTok{tokens}\NormalTok{(data\_corpus\_irishbudget2010, }\AttributeTok{remove\_punct =} \ConstantTok{TRUE}\NormalTok{) }\SpecialCharTok{\%\textgreater{}\%}
\FunctionTok{tokens\_tolower}\NormalTok{() }\SpecialCharTok{\%\textgreater{}\%}
\FunctionTok{tokens\_remove}\NormalTok{(}\FunctionTok{stopwords}\NormalTok{(}\StringTok{"english"}\NormalTok{), }\AttributeTok{padding =} \ConstantTok{FALSE}\NormalTok{) }\SpecialCharTok{\%\textgreater{}\%}
\FunctionTok{fcm}\NormalTok{(}\AttributeTok{context =} \StringTok{"window"}\NormalTok{, }\AttributeTok{window =} \DecValTok{5}\NormalTok{, }\AttributeTok{tri =} \ConstantTok{FALSE}\NormalTok{)}
\CommentTok{\# choose 30 most frequency features}
\NormalTok{topfeats }\OtherTok{\textless{}{-}} \FunctionTok{names}\NormalTok{(}\FunctionTok{topfeatures}\NormalTok{(examplefcm, }\DecValTok{30}\NormalTok{))}
\CommentTok{\# select the top 30 features only, plot the network}
\FunctionTok{set.seed}\NormalTok{(}\DecValTok{100}\NormalTok{)}
\FunctionTok{textplot\_network}\NormalTok{(}\FunctionTok{fcm\_select}\NormalTok{(examplefcm, topfeats), }\AttributeTok{min\_freq =} \FloatTok{0.8}\NormalTok{)}
\end{Highlighting}
\end{Shaded}

On 2023-02-08, this code snippet is not executable with the current
version of \texttt{quanteda} (3.2.4). It is possible to install the
``period appropriate'' version of \texttt{quanteda} (1.3.4) using
\texttt{remotes} on the current version of R (4.2.2). And indeed, the
above code snippet can still be executed.

\begin{Shaded}
\begin{Highlighting}[]
\NormalTok{remotes}\SpecialCharTok{::}\FunctionTok{install\_version}\NormalTok{(}\StringTok{"quanteda"}\NormalTok{, }\AttributeTok{version =} \StringTok{"1.3.4"}\NormalTok{)}
\end{Highlighting}
\end{Shaded}

The issue is that installing \texttt{quanteda} 1.3.4 this way installs
the latest dependencies from CRAN. \texttt{quanteda} 1.3.4 uses a
deprecated (but not yet removed) function of \texttt{Matrix}
(\texttt{as(\textless{}dgTMatrix\textgreater{},\ "dgCMatrix")}). If this
function were removed in the future, the above code snippet would not be
executable anymore.

Using \texttt{rang}, one can query the version of \texttt{quanteda} on
2018-10-06 and create a Docker container with all the ``period
appropriate'' dependencies. Here, the \texttt{rstudio} Rocker image is
selected.

\begin{Shaded}
\begin{Highlighting}[]
\FunctionTok{library}\NormalTok{(rang)}
\NormalTok{graph }\OtherTok{\textless{}{-}} \FunctionTok{resolve}\NormalTok{(}\AttributeTok{pkgs =} \StringTok{"quanteda"}\NormalTok{,}
                 \AttributeTok{snapshot\_date =} \StringTok{"2018{-}10{-}06"}\NormalTok{,}
                 \AttributeTok{os =} \StringTok{"ubuntu{-}18.04"}\NormalTok{)}
\FunctionTok{dockerize}\NormalTok{(graph, }\AttributeTok{output\_dir =} \StringTok{"quanteda\_docker"}\NormalTok{,}
          \AttributeTok{image =} \StringTok{"rstudio"}\NormalTok{)}
\end{Highlighting}
\end{Shaded}

The above code snippet can be executed with the generated container
without any problem Fig~\ref{fig1}.

% Place figure captions after the first paragraph in which they are cited.
\begin{figure}[!h]
\caption{The code snippet running in a R 3.5.1 container created with rang.}
\label{fig1}
\end{figure}

\hypertarget{psychological-science}{%
\subsection{Psychological Science}\label{psychological-science}}

Crüwell et al. \citep{cruewell:2023:WB} evaluate the computational
reproducibility of 14 articles published in \emph{Psyhocological
Science}. Among these articles, the paper by Hilgard et al.
\citep{hilgard:2019:NEG} has been rated as having ``package dependency
issues''.

All data and computer code are available from GitHub with the last
commit on 2019-01-17 \footnote{\url{https://github.com/Joe-Hilgard/vvg-2d4d}}.
The R code contains a list of R packages used in the project as
\texttt{library()} statements, including an R package on GitHub that is
written by the main author of that paper. However, we identified one
package (\texttt{compute.es}) that was not written in those
\texttt{library()} statements but used with the namespace operator,
i.e.~\texttt{compute.es::tes()}. This undocumented package can be
detected by \texttt{renv::dependencies()}, which is the provider of the
scanning function of \texttt{rang}.

Based on the above information, one can run \texttt{resolve()} to obtain
the dependency graph of all R packages on 2019-01-17.

\begin{Shaded}
\begin{Highlighting}[]
\DocumentationTok{\#\# scan all packages}
\NormalTok{r\_pkgs }\OtherTok{\textless{}{-}} \FunctionTok{as\_pkgrefs}\NormalTok{(}\StringTok{"vvg{-}2d4d"}\NormalTok{)}
\DocumentationTok{\#\# replace cran::hilgard with Github}
\NormalTok{r\_pkgs[r\_pkgs }\SpecialCharTok{==} \StringTok{"cran::hilhard"}\NormalTok{] }\OtherTok{\textless{}{-}} \StringTok{"Joe{-}Hilgard/hilgard"}
\NormalTok{graph }\OtherTok{\textless{}{-}} \FunctionTok{resolve}\NormalTok{(r\_pkgs, }\AttributeTok{snapshot\_date =} \StringTok{"2019{-}01{-}17"}\NormalTok{)}
\end{Highlighting}
\end{Shaded}

When running \texttt{dockerize()}, one can take advantage of the
\texttt{materials\_dir} parameter to transfer the shared materials from
Hilgard et al. \citep{hilgard:2019:NEG} into the Docker image.

\begin{Shaded}
\begin{Highlighting}[]
\FunctionTok{dockerize}\NormalTok{(graph, }\StringTok{"hilgard"}\NormalTok{, }\AttributeTok{materials\_dir =} \StringTok{"vvg{-}2d4d"}\NormalTok{, }\AttributeTok{cache =} \ConstantTok{TRUE}\NormalTok{)}
\end{Highlighting}
\end{Shaded}

We then built the Docker and launch a Docker container. For this
container, we changed the entry point from R to bash so that the
container goes to the Linux command shell instead.

\begin{Shaded}
\begin{Highlighting}[]
\NormalTok{cd hilgard}
\NormalTok{docker build }\SpecialCharTok{{-}}\NormalTok{t hilgard .}
\NormalTok{docker run }\SpecialCharTok{{-}{-}}\NormalTok{rm }\SpecialCharTok{{-}{-}}\NormalTok{name }\StringTok{"hilgardcontainer"} \SpecialCharTok{{-}{-}}\NormalTok{entrypoint bash }\SpecialCharTok{{-}}\NormalTok{ti hilgard}
\end{Highlighting}
\end{Shaded}

Inside the container, the materials are located in the
\texttt{materials} directory. We used the following shell script to test
the reproducibility of all R scripts.

\begin{Shaded}
\begin{Highlighting}[]
\BuiltInTok{cd}\NormalTok{ materials}
\VariableTok{rfiles}\OperatorTok{=}\VariableTok{(}\NormalTok{0\_data\_aggregation.R 1\_data\_cleaning.R 2\_analysis.R 3\_plotting.R}\VariableTok{)}
\ControlFlowTok{for}\NormalTok{ i }\KeywordTok{in} \VariableTok{$\{rfiles}\OperatorTok{[@]}\VariableTok{\}}
\ControlFlowTok{do}
    \ExtensionTok{Rscript} \VariableTok{$i}
    \VariableTok{code}\OperatorTok{=}\VariableTok{$?}
    \ControlFlowTok{if} \BuiltInTok{[} \VariableTok{$code} \OtherTok{!=}\NormalTok{ 0 }\BuiltInTok{]}
    \ControlFlowTok{then}
        \BuiltInTok{exit}\NormalTok{ 1}
    \ControlFlowTok{fi}
\ControlFlowTok{done}
\end{Highlighting}
\end{Shaded}

All R scripts ran fine inside the container and the figures generated
are the same as the ones in Hilgard et al. \citep{hilgard:2019:NEG}.

\hypertarget{political-analysis}{%
\subsection{Political Analysis}\label{political-analysis}}

The study by Trisovic et al. \citep{trisovic:2022} evaluates the
reproducibility of R scripts shared on Dataverse. They found that 75\%
of R scripts cannot be successfully executed. Among these failed R
scripts is an R script shared by Beck \citep{beck:2019:EGD}.

This R script has been ``rescued'' by the author of the R package
\texttt{groundhog} \citep{groundhogrpkg}, as demonstrated in a blog post
\footnote{\url{http://datacolada.org/100}}. We were wondering if
\texttt{rang} can also be used to ``rescue'' the concerned R script. The
date of the R script, as indicated on Dataverse, is 2018-12-12. This
date is used as the snapshot date.

\begin{Shaded}
\begin{Highlighting}[]
\DocumentationTok{\#\# as\_pkgrefs is automatically run in this case}
\NormalTok{graph }\OtherTok{\textless{}{-}} \FunctionTok{resolve}\NormalTok{(}\StringTok{"nathaniel"}\NormalTok{, }\AttributeTok{snapshot\_date =} \StringTok{"2018{-}12{-}12"}\NormalTok{)}
\FunctionTok{dockerize}\NormalTok{(graph, }\AttributeTok{output\_dir =} \StringTok{"nat"}\NormalTok{, }\AttributeTok{materials\_dir =} \StringTok{"nathaniel"}\NormalTok{)}
\end{Highlighting}
\end{Shaded}

\begin{Shaded}
\begin{Highlighting}[]
\NormalTok{cd nat}
\NormalTok{docker build }\SpecialCharTok{{-}}\NormalTok{t nat .}
\NormalTok{docker run }\SpecialCharTok{{-}{-}}\NormalTok{rm }\SpecialCharTok{{-}{-}}\NormalTok{name }\StringTok{"natcontainer"} \SpecialCharTok{{-}{-}}\NormalTok{entrypoint bash }\SpecialCharTok{{-}}\NormalTok{ti nat}
\end{Highlighting}
\end{Shaded}

Inside the container

\begin{Shaded}
\begin{Highlighting}[]
\NormalTok{cd materials}
\NormalTok{Rscript fn\_5.R}
\end{Highlighting}
\end{Shaded}

The same file can thus also be ``rescued'' by \texttt{rang}.

\hypertarget{recover-a-removed-r-package-maxent}{%
\subsection{Recover a removed R package:
maxent}\label{recover-a-removed-r-package-maxent}}

The R package \texttt{maxent} introduces a machine learning algorithm
with a small memory footprint and was available on CRAN until 2019. A
software paper was published by the original authors in 2012
\citep{jurka:2012}. The R package was also used in some subsequent
automated content analytic papers \citep[e.g.][]{loercher:2017:D}.
Despite the covert editing of the package by a staffer of CRAN
\footnote{\url{https://github.com/cran/maxent/commit/9d46c6aad27a1f41a78907b170ddd9a586192be9}},
the package was removed from CRAN in 2019 \footnote{This is the check
  log:
  \url{https://cran-archive.r-project.org/web/checks/2019/2019-03-05_check_results_maxent.html}
  Here, we said the package \texttt{maxent} was ``removed from CRAN'' as
  per what the webpage
  (\url{https://cran.r-project.org/web/packages/maxent/index.html}) is
  written. However, the terminology as stated in the
  \href{https://cran.r-project.org/web/packages/policies.html\#Source-packages}{CRAN
  Policies} is ``archived'', quote: ``Packages will not normally be
  removed from CRAN: however, they may be archived, including at the
  maintainer's request.'' As all CRAN packages have all the submitted
  versions archived, we find this terminology confusing. Therefore, we
  use ``removed'' packages throughout this paper to indicate packages
  that cannot be installed by the usual method,
  i.e.~\texttt{install.packages} but still have the old versions
  archived on CRAN.}. We attempted to install the second last (the
original submitted version) and last (with covert editing) versions of
\texttt{maxent} on R 4.2.2. Both of them didn't work.

Using \texttt{rang}, we are able to reconstruct a computational
environment with R 2.15.0 (2012-03-30) to run all code snippets
published in Jurka \citep{jurka:2012} \footnote{On an interesting
  historical side note: The original paper reported ---based on a
  benchmark--- that ``the algorithm is very fast; \texttt{maxent} uses
  only 135.4 megabytes of RAM and finishes in 53.3 seconds.'' On a
  modest computer in 2023 with a dockerized R 2.15.0, the benchmark
  finishes in 4 seconds.}. For removed CRAN packages, we strongly
recommend querying the Github read-only mirror of CRAN instead
(https://github.com/cran). It is because in this way, the resolved
system requirements have a higher chance of being correct.

\begin{Shaded}
\begin{Highlighting}[]
\NormalTok{maxent }\OtherTok{\textless{}{-}} \FunctionTok{resolve}\NormalTok{(}\StringTok{"cran/maxent"}\NormalTok{, }\StringTok{"2012{-}06{-}10"}\NormalTok{)}
\FunctionTok{dockerize}\NormalTok{(maxent, }\StringTok{"maxentdir"}\NormalTok{, }\AttributeTok{cache =} \ConstantTok{TRUE}\NormalTok{)}
\end{Highlighting}
\end{Shaded}

\hypertarget{recover-a-removed-r-package-ptproc}{%
\subsection{Recover a removed R package:
ptproc}\label{recover-a-removed-r-package-ptproc}}

The software paper of the R package \texttt{ptproc} was published in
2003 and introduced multidimensional point process models
\citep{peng:2003:MDP}. But the package has been removed from CRAN for
over a decade (at least). The only release on CRAN was on 2002-10-10.
The package is still listed in the ``Handling and Analyzing
Spatio-Temporal Data'' CRAN Task View \footnote{\url{https://cran.r-project.org/web/views/SpatioTemporal.html}}
despite being uninstallable without modification on any modern R system
(see below). As of writing, the package, as a tarball file (tar.gz), is
still downloadable from the original author's website \footnote{\url{https://www.biostat.jhsph.edu/~rpeng/software/}}.

Even with this over-a-decade removal and new packages with similar
functionalities have been created, there is evidence that
\texttt{ptproc} is still being sought for. As late as 2017, there are
blog posts on how to install the long obsolete package on modern
versions of R \footnote{\url{https://blog.mathandpencil.com/installing-ptproc-on-osx}
  and
  \url{https://tomaxent.com/2017/03/16/Installing-ptproc-on-Ubuntu-16-04-LTS/}}.
The package is extremely challenging to install on a modern R system
because the package was written before the introduction of name space
management in R 1.7.0 \citep{RN-2003-001}. In other words, the available
tarball files from the original author's website and CRAN do not contain
a \texttt{NAMESPACE} file like all other modern R packages do.

The oldest version of R that \texttt{rang} can support, as of writing,
is R 1.3.1. \texttt{rang} is probably the only solution available that
can support the 1.x series of R (i.e.~before 2004-10-04). Similar to the
case of \texttt{maxent} above, a \texttt{Dockerfile} to assemble a
Docker image with \texttt{ptproc} installed can be generated with two
lines of code.

\begin{Shaded}
\begin{Highlighting}[]
\NormalTok{graph }\OtherTok{\textless{}{-}} \FunctionTok{resolve}\NormalTok{(}\StringTok{"ptproc"}\NormalTok{, }\AttributeTok{snapshot\_date =} \StringTok{"2004{-}07{-}01"}\NormalTok{)}
\FunctionTok{dockerize}\NormalTok{(graph, }\StringTok{"\textasciitilde{}/dev/misc/ptproc"}\NormalTok{, }\AttributeTok{cache =} \ConstantTok{TRUE}\NormalTok{)}
\end{Highlighting}
\end{Shaded}

Suppose we have an R script, extracted from Peng \citep{peng:2003:MDP},
called ``peng.R'' like this:

\begin{Shaded}
\begin{Highlighting}[]
\FunctionTok{require}\NormalTok{(ptproc)}

\FunctionTok{set.seed}\NormalTok{(}\DecValTok{1000}\NormalTok{)}
\NormalTok{x }\OtherTok{\textless{}{-}} \FunctionTok{cbind}\NormalTok{(}\FunctionTok{runif}\NormalTok{(}\DecValTok{100}\NormalTok{), }\FunctionTok{runif}\NormalTok{(}\DecValTok{100}\NormalTok{), }\FunctionTok{runif}\NormalTok{(}\DecValTok{100}\NormalTok{))}
\NormalTok{hPois.cond.int }\OtherTok{\textless{}{-}} \ControlFlowTok{function}\NormalTok{(params, eval.pts, }\AttributeTok{pts =} \ConstantTok{NULL}\NormalTok{,}
                           \AttributeTok{data =} \ConstantTok{NULL}\NormalTok{, }\AttributeTok{TT =} \ConstantTok{NULL}\NormalTok{) \{}
\NormalTok{    mu }\OtherTok{\textless{}{-}}\NormalTok{ params[}\DecValTok{1}\NormalTok{]}
    \ControlFlowTok{if}\NormalTok{(}\FunctionTok{is.null}\NormalTok{(TT))}
        \FunctionTok{rep}\NormalTok{(mu, }\FunctionTok{nrow}\NormalTok{(eval.pts))}
    \ControlFlowTok{else}\NormalTok{ \{}
\NormalTok{        vol }\OtherTok{\textless{}{-}} \FunctionTok{prod}\NormalTok{(}\FunctionTok{apply}\NormalTok{(TT, }\DecValTok{2}\NormalTok{, diff))}
\NormalTok{        mu }\SpecialCharTok{*}\NormalTok{ vol}
\NormalTok{    \}}
\NormalTok{\}}
\NormalTok{ppm }\OtherTok{\textless{}{-}} \FunctionTok{ptproc}\NormalTok{(}\AttributeTok{pts =}\NormalTok{ x, }\AttributeTok{cond.int =}\NormalTok{ hPois.cond.int, }\AttributeTok{params =} \DecValTok{50}\NormalTok{,}
              \AttributeTok{ranges =} \FunctionTok{cbind}\NormalTok{(}\FunctionTok{c}\NormalTok{(}\DecValTok{0}\NormalTok{,}\DecValTok{1}\NormalTok{), }\FunctionTok{c}\NormalTok{(}\DecValTok{0}\NormalTok{,}\DecValTok{1}\NormalTok{), }\FunctionTok{c}\NormalTok{(}\DecValTok{0}\NormalTok{,}\DecValTok{1}\NormalTok{)))}
\NormalTok{fit }\OtherTok{\textless{}{-}} \FunctionTok{ptproc.fit}\NormalTok{(ppm, }\AttributeTok{optim.control =} \FunctionTok{list}\NormalTok{(}\AttributeTok{trace =} \DecValTok{2}\NormalTok{), }\AttributeTok{method =} \StringTok{"BFGS"}\NormalTok{)}
\FunctionTok{summary}\NormalTok{(fit)}
\end{Highlighting}
\end{Shaded}

One can integrate \texttt{rang} into a BASH script to completely
automate the batch execution of the above R script.

\begin{Shaded}
\begin{Highlighting}[]
\ExtensionTok{Rscript} \AttributeTok{{-}e} \StringTok{"require(rang); dockerize(resolve(\textquotesingle{}ptproc\textquotesingle{}, \textquotesingle{}2004{-}07{-}01\textquotesingle{}),}
\StringTok{\textquotesingle{}pengdocker\textquotesingle{}, cache = TRUE)"}
\ExtensionTok{docker}\NormalTok{ build }\AttributeTok{{-}t}\NormalTok{ pengimg ./pengdocker}
\CommentTok{\#\# launching a container in daemon mode {-}d}
\ExtensionTok{docker}\NormalTok{ run }\AttributeTok{{-}d} \AttributeTok{{-}{-}rm} \AttributeTok{{-}{-}name} \StringTok{"pengcontainer"} \AttributeTok{{-}ti}\NormalTok{ pengimg}
\ExtensionTok{docker}\NormalTok{ cp peng.R pengcontainer:/peng.R}
\ExtensionTok{docker}\NormalTok{ exec pengcontainer R CMD BATCH peng.R}
\ExtensionTok{docker}\NormalTok{ exec pengcontainer cat peng.Rout}
\ExtensionTok{docker}\NormalTok{ cp pengcontainer:/peng.Rout peng.Rout}
\ExtensionTok{docker}\NormalTok{ stop pengcontainer}
\end{Highlighting}
\end{Shaded}

The file \texttt{peng.Rout} contains the execution results of the script
from inside the Docker container. As the random seed was preserved by
the original author \citep{peng:2003:MDP}, the above BASH script can
perfectly reproduce the analysis \footnote{It is also important to note
  that the random number generator (RNG) of R has been changed several
  times over the course of the development. In this case, we are using
  the same generation of RNG as Peng \citep{peng:2003:MDP}.}.

\hypertarget{recover-a-removed-bioconductor-package}{%
\subsection{Recover a removed Bioconductor
package}\label{recover-a-removed-bioconductor-package}}

Similar to CRAN, packages can also be removed over time from
Bioconductor. The Bioconductor package \texttt{Sushi} has been
deprecated by the original authors and is removed from Bioconductor
version 3.16 (2022-11-02). \texttt{Sushi} is a data visualization tool
for genomic data and was used in many online tutorials and scientific
papers, including the original paper announcing the package by the
original authors \citep{phanstiel:2014:S}.

\texttt{rang} has native support for Bioconductor packages since version
0.2. We obtained the R script \texttt{"PaperFigure.R"} from the Github
repository of \texttt{Sushi} \footnote{\url{https://github.com/PhanstielLab/Sushi/blob/master/vignettes/PaperFigure.R}},
which generates the figure in the original paper
\citep{phanstiel:2014:S}. Similar to the above case of \texttt{ptproc},
we made a completely automated BASH script to run
\texttt{"PaperFigure.R"} and get the generated figure out of the
container (Fig~\ref{fig2}). We made no modification to
\texttt{"PaperFigure.R"}.

\begin{Shaded}
\begin{Highlighting}[]
\ExtensionTok{Rscript} \AttributeTok{{-}e} \StringTok{"require(rang); dockerize(resolve(\textquotesingle{}Sushi\textquotesingle{}, \textquotesingle{}2014{-}06{-}05\textquotesingle{}),}
\StringTok{\textquotesingle{}sushidocker\textquotesingle{}, no\_rocker = TRUE, cache = TRUE)"}
\ExtensionTok{docker}\NormalTok{ build }\AttributeTok{{-}t}\NormalTok{ sushiimg ./sushidocker}
\ExtensionTok{docker}\NormalTok{ run }\AttributeTok{{-}d} \AttributeTok{{-}{-}rm} \AttributeTok{{-}{-}name} \StringTok{"sushicontainer"} \AttributeTok{{-}ti}\NormalTok{ sushiimg}
\ExtensionTok{docker}\NormalTok{ cp PaperFigure.R sushicontainer:/PaperFigure.R}
\ExtensionTok{docker}\NormalTok{ exec sushicontainer mkdir vignettes}
\ExtensionTok{docker}\NormalTok{ exec sushicontainer R CMD BATCH PaperFigure.R}
\ExtensionTok{docker}\NormalTok{ cp sushicontainer:/vignettes/Figure\_1.pdf sushi\_figure1.pdf}
\ExtensionTok{docker}\NormalTok{ stop sushicontainer}
\end{Highlighting}
\end{Shaded}

% Place figure captions after the first paragraph in which they are cited.
\begin{figure}[!h]
\caption{The figure from the batch execution of PaperFigure.R inside a Docker container generated by rang}
\label{fig2}
\end{figure}

\hypertarget{preparing-research-compendia-with-long-term-computational-reproducibility}{%
\section{Preparing research compendia with long-term computational
reproducibility}\label{preparing-research-compendia-with-long-term-computational-reproducibility}}

The above six examples show how powerful \texttt{rang} is to reconstruct
tricky computational environments which have not been completely
declared in the literature. Although we position \texttt{rang} mostly as
an archaeological tool, we think that \texttt{rang} can also be used to
prepare research compendia of current research. We can't predict the
future but research compendia generated by \texttt{rang} would probably
have long-term computational reproducibility.

To demonstrate this point, we took the recent paper by Oser et al.
\citep{oser:2022:HPE}. This paper was selected because 1) the paper was
published in \emph{Political Communication}, a high impact journal that
awards Open Science Badges; 2) shared data and R code are available; and
most importantly, 3) the shared R code is well-written. In the
repository of this paper, we based on the materials shared by Oser et
al. \citep{oser:2022:HPE} and prepared a research compendium that should
have long-term computational reproducibility. The research compendium is
similar to the Executable Compendium suggested by the Turing way.

The preparation of the research compendium is easy as \texttt{rang} can
scan a materials directory for all R packages used \footnote{We detected
  a minor issue in the code base that an undeclared Github package is
  used. But it can be easily solved, as in the Psychological Science
  example above.}.

\begin{Shaded}
\begin{Highlighting}[]
\FunctionTok{require}\NormalTok{(rang)}
\DocumentationTok{\#\# meta{-}analysis is the directory of all shared materials}
\NormalTok{cran\_pkgs }\OtherTok{\textless{}{-}} \FunctionTok{as\_pkgrefs}\NormalTok{(}\StringTok{"meta{-}analysis"}\NormalTok{) }

\DocumentationTok{\#\# dmetar is an undeclared github package: MathiasHarrer/dmetar}
\NormalTok{cran\_pkgs[cran\_pkgs }\SpecialCharTok{==} \StringTok{"cran::dmetar"}\NormalTok{] }\OtherTok{\textless{}{-}} \StringTok{"MathiasHarrer/dmetar"}
\NormalTok{x }\OtherTok{\textless{}{-}} \FunctionTok{resolve}\NormalTok{(cran\_pkgs, }\StringTok{"2021{-}08{-}11"}\NormalTok{, }\AttributeTok{verbose =} \ConstantTok{TRUE}\NormalTok{)}
\DocumentationTok{\#\#print(x, all\_pkgs = TRUE)}
\FunctionTok{dockerize}\NormalTok{(x, }\StringTok{"oserdocker"}\NormalTok{, }\AttributeTok{materials\_dir =} \StringTok{"meta{-}analysis"}\NormalTok{, }\AttributeTok{cache =} \ConstantTok{TRUE}\NormalTok{)}
\end{Highlighting}
\end{Shaded}

The above R script is saved as \texttt{oser.R}. The central piece of the
executable compendium is the \texttt{Makefile} \citep{baker:2020:UGM}.

\begin{Shaded}
\begin{Highlighting}[]
\DataTypeTok{output\_file}\CharTok{=}\StringTok{reproduced.html}
\DataTypeTok{r\_cmd }\CharTok{=}\StringTok{ "rmarkdown::render(\textquotesingle{}materials/README.Rmd\textquotesingle{},}\CharTok{\textbackslash{}}
\StringTok{output\_file = \textquotesingle{}}\CharTok{$\{}\DataTypeTok{output\_file}\CharTok{\}}\StringTok{\textquotesingle{})"}
\DataTypeTok{handle}\CharTok{=}\StringTok{oser}
\DataTypeTok{local\_file}\CharTok{=$\{}\DataTypeTok{handle}\CharTok{\}}\StringTok{\_README.html}

\OtherTok{.PHONY:}\DataTypeTok{ all resolve build render export rebuild}

\DecValTok{all:}\DataTypeTok{ resolve build render}
\NormalTok{    echo }\StringTok{"finished"}
\DecValTok{resolve:}
\NormalTok{    Rscript }\CharTok{$\{}\DataTypeTok{handle}\CharTok{\}}\NormalTok{.R}
\DecValTok{build:}\DataTypeTok{ }\CharTok{$\{}\DataTypeTok{handle}\CharTok{\}}\DataTypeTok{docker}
\NormalTok{    docker build {-}t }\CharTok{$\{}\DataTypeTok{handle}\CharTok{\}}\NormalTok{img }\CharTok{$\{}\DataTypeTok{handle}\CharTok{\}}\NormalTok{docker}
\DecValTok{render:}
\NormalTok{    docker run {-}d {-}{-}rm {-}{-}name }\StringTok{"}\CharTok{$\{}\DataTypeTok{handle}\CharTok{\}}\StringTok{container"}\NormalTok{ {-}ti }\CharTok{$\{}\DataTypeTok{handle}\CharTok{\}}\NormalTok{img}
\NormalTok{    docker exec }\CharTok{$\{}\DataTypeTok{handle}\CharTok{\}}\NormalTok{container Rscript {-}e }\CharTok{$\{}\DataTypeTok{r\_cmd}\CharTok{\}}
\NormalTok{    docker cp }\CharTok{$\{}\DataTypeTok{handle}\CharTok{\}}\NormalTok{container:/materials/}\CharTok{$\{}\DataTypeTok{output\_file}\CharTok{\}} \CharTok{$\{}\DataTypeTok{local\_file}\CharTok{\}}
\NormalTok{    docker stop }\CharTok{$\{}\DataTypeTok{handle}\CharTok{\}}\NormalTok{container}
\DecValTok{export:}
\NormalTok{    docker save }\CharTok{$\{}\DataTypeTok{handle}\CharTok{\}}\NormalTok{img | gzip \textgreater{} }\CharTok{$\{}\DataTypeTok{handle}\CharTok{\}}\NormalTok{img.tar.gz}
\DecValTok{rebuild:}\DataTypeTok{ }\CharTok{$\{}\DataTypeTok{handle}\CharTok{\}}\DataTypeTok{img.tar.gz}
\NormalTok{    docker load \textless{} }\CharTok{$\{}\DataTypeTok{handle}\CharTok{\}}\NormalTok{img.tar.gz}
\end{Highlighting}
\end{Shaded}

With this \texttt{Makefile}, one can create the \texttt{Dockerfile} with
\texttt{make\ resolve}, build the Docker image with
\texttt{make\ build}, render the RMarkdown file inside the container
with \texttt{make\ render}, export the built Docker image with
\texttt{make\ export}, and rebuild the exported Docker image with
\texttt{make\ rebuild}.

The structure of the entire executable compendium looks like this:

\begin{Shaded}
\begin{Highlighting}[]
\ExtensionTok{Makefile}
\ExtensionTok{oser.R}
\ExtensionTok{meta{-}analysis/}
\ExtensionTok{README.md}
\ExtensionTok{oserdocker/}
\ExtensionTok{oserimg.tar.gz}
\end{Highlighting}
\end{Shaded}

In this executable compendium, only the first four elements are
essential. The directory \texttt{oserdocker} (116 MB) contains cached R
packages, a \texttt{Dockerfile}, and a verbatim copy of the directory
\texttt{meta-analysis/} to be transferred into the Docker image. That
can be regenerated by running \texttt{make\ resolve}. However, having
this directory preserved insures against the situations that some R
packages used in the project were no longer available or any of the
information providers used by \texttt{rang} for resolving the dependency
relationships were not available. (Or in the rare circumstance of
\texttt{rang} is no longer available.)

\texttt{oserimg.tar.gz} (667 MB) is a backup copy of the Docker image.
This can be regenerated by running \texttt{make\ export}. Preserving
this file insures against all the situations mentioned above, but also
the situations of Docker Hub (the hosting service provided by Docker for
base images such as Rocker) and the software repositories used by the
dockerized operating system being not available. When
\texttt{oserimg.tar.gz} is available, it is possible to run
\texttt{make\ rebuild} and \texttt{make\ render} even without internet
access (provided that Docker and \texttt{make} have been installed
before). Of course, there is still an extremely rare situation where
Docker (the program) itself is no longer available \footnote{We can't
  imagine a world without \texttt{Make}, a tool that has been available
  since 1976.}. However, it is possible to convert the image file for
use on other containerization solutions such as Singularity \footnote{\url{https://docs.sylabs.io/guides/3.0/user-guide/singularity_and_docker.html}},
if Docker is really not available anymore.

Sharing of research artifacts less than 1G is not as challenging as it
used to be. Zenodo, for example, allows the sharing of 50G of files.
Therefore, sharing of the last two components of the executable
compendium prepared with \texttt{rang} is at least possible on Zenodo
\footnote{The complete version of the executable compendium is available
  on Zenodo: \url{https://doi.org/10.5281/zenodo.7708417}}. However, for
data repositories with more restrictions on data size, sharing the
executable compendium without the last two parts could be considered
sufficient. For that, run \texttt{make} will make the default target
\texttt{all} and generate all the things needed for reproducing the
analysis inside a container.

The above \texttt{Makefile} is general enough that one can reuse it by
just modifying how the R scripts (the \texttt{r\_cmd} variable) in the
\texttt{materials} directory are executed. This can be a starting point
of a standard executable compendium format.

\hypertarget{concluding-remarks}{%
\section{Concluding remarks}\label{concluding-remarks}}

This paper presents \texttt{rang}, a solution to (re)construct R
computational environments based on Docker. As the six examples in
Section 3 show, \texttt{rang} can be used archaeologically to rerun old
code, many of them not executable without the analytic and
reconstruction processes facilitated by \texttt{rang}. These
retrospective use cases demonstrate how versatile \texttt{rang} is.
\texttt{rang} is also helpful for prospective usage, as demonstrated in
Section 4 whereby an executable compendium is created.

There are still many features that we did not mention in this paper.
\texttt{rang} is built with interoperability in mind. As of writing,
\texttt{rang} is interoperable with existing R packages such as
\texttt{renv} and R built-in \texttt{sessionInfo()}. Also, the
\texttt{rang} object can be used for network analysis with R packages
such as \texttt{igraph}.

Computational reproducibility is a complex topic and as in all of these
complex topics, there is no silver bullet \citep{canon:2019:CPR}. All
solutions have their trade-offs. The (re)construction process based on
\texttt{rang} takes notably more time than other solutions because all
packages are compiled from source. \texttt{rang} trades computational
efficiency of this often one-off (re)constructing process for
correctness, backward compatibility and independence from any commercial
backups of software repositories such as MRAN. There are also other
limitations. In the Vignette of \texttt{rang}
(\url{https://cran.r-project.org/web/packages/rang/vignettes/faq.html}),
we list all of these limitations as well as possible mitigation.


\nolinenumbers
\renewcommand\refname{References}
  \bibliography{paper.bib}

\end{document}
